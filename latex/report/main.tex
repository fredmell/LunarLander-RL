%%%%%%%% ICML 2020 EXAMPLE LATEX SUBMISSION FILE %%%%%%%%%%%%%%%%%

\documentclass{article}

% Recommended, but optional, packages for figures and better typesetting:
\usepackage{microtype}
\usepackage{graphicx}
\usepackage{subfigure}
\usepackage{booktabs} % for professional tables

% hyperref makes hyperlinks in the resulting PDF.
% If your build breaks (sometimes temporarily if a hyperlink spans a page)
% please comment out the following usepackage line and replace
% \usepackage{icml2020} with \usepackage[nohyperref]{icml2020} above.
\usepackage{hyperref}

% Attempt to make hyperref and algorithmic work together better:
\newcommand{\theHalgorithm}{\arabic{algorithm}}

% Use the following line for the initial blind version submitted for review:
% \usepackage{icml2020}

% If accepted, instead use the following line for the camera-ready submission:
\usepackage[accepted]{icml2020}

\usepackage{mathtools}
\usepackage{amsmath}

% The \icmltitle you define below is probably too long as a header.
% Therefore, a short form for the running title is supplied here:
\icmltitlerunning{Solving Lunar Lander with a DQN}

\begin{document}

\twocolumn[
\icmltitle{Solving Lunar Lander with a DQN}

% It is OKAY to include author information, even for blind
% submissions: the style file will automatically remove it for you
% unless you've provided the [accepted] option to the icml2020
% package.

% List of affiliations: The first argument should be a (short)
% identifier you will use later to specify author affiliations
% Academic affiliations should list Department, University, City, Region, Country
% Industry affiliations should list Company, City, Region, Country

% You can specify symbols, otherwise they are numbered in order.
% Ideally, you should not use this facility. Affiliations will be numbered
% in order of appearance and this is the preferred way.
% \icmlsetsymbol{equal}{*}

\begin{icmlauthorlist}
\icmlauthor{Frederik J. Mellbye}{to}
\end{icmlauthorlist}

\icmlaffiliation{to}{Institute for Computational and Mathematical Engineering, Stanford University, Stanford, USA}

\icmlcorrespondingauthor{Frederik J. Mellbye}{frederme@stanford.edu}

% You may provide any keywords that you
% find helpful for describing your paper; these are used to populate
% the "keywords" metadata in the PDF but will not be shown in the document
\icmlkeywords{Machine Learning, ICML}

\vskip 0.3in
]

% this must go after the closing bracket ] following \twocolumn[ ...

% This command actually creates the footnote in the first column
% listing the affiliations and the copyright notice.
% The command takes one argument, which is text to display at the start of the footnote.
% The \icmlEqualContribution command is standard text for equal contribution.
% Remove it (just {}) if you do not need this facility.

\printAffiliationsAndNotice{}  % leave blank if no need to mention equal contribution
% \printAffiliationsAndNotice{\icmlEqualContribution} % otherwise use the standard text.

\begin{abstract}
This document provides a basic paper template and submission guidelines.
Abstracts must be a single paragraph, ideally between 4--6 sentences long.
Gross violations will trigger corrections at the camera-ready phase.
\end{abstract}

\section{Introduction}
\label{introduction}

Game playing important step in development of RL.

Lunar Lander somewhat complicated in the following sense: Simple game inputs (thrust L/R/D), but need to perform multiple tasks to land successfully. Need to keep lander somewhat level, turn towards landing pad, smooth braking towards landing pad, understanding that no thrust is needed to finally land.

Lunar Lander simple control environemnt, very applicable to engineering.

\section{Background}
\label{background}

Nature paper. \cite{mnih2015humanlevel}

Double DQN paper. \cite{Hasselt2016DeepRL}

\section{Approach}
\label{approach}
Parameters Table~\ref{tbl:params}.

Similar approach to \cite{Mnih2013PlayingAW}. Takes game state directly as input instead of screen pixel values, so does not consider vision.

$\epsilon$-greedy: Linear annealing.

\subsection{Deep Q-learning}

\subsection{Double DQN}

\subsection{Prioritized experience replay}

\begin{algorithm}[tb]
   \caption{Double Deep Q-learning.}
   \label{alg:ddqn}
\begin{algorithmic}[1]
  \FOR{$\text{episode} = 1, \hdots, M$}
    \STATE Observe $s_1$
    \FOR{$t = 1, \hdots, T$}
      \STATE Select action
      \begin{align*}
        a_t = \begin{cases}
          \text{Random} & \text{with probability } \varepsilon \\
          \arg \max_a Q(s_t, a; \theta) & \text{otherwise}
        \end{cases}
      \end{align*}
      \STATE Perform action $a_t$ and observe $r_t$ and $s_{t+1}$
      \STATE Store experience tuple $(s_{t}, a_{t}, r_{t}, s_{t+1})$ in $\mathcal{D}$
      \STATE Sample random minibatch of transitions $(s_j, a_j, r_j, s_{j+1})$
      \STATE Set targets $y_j = $
      \begin{align*}
        \begin{cases}
          r_j & \text{If episode was terminal} \\
          r_j + \gamma \max_{a'} Q(s_{j+1}, a'; \theta^{-}) & \text{otherwise}
        \end{cases}
      \end{align*}
      \STATE Perform one epoch of gradient descent with loss $(y_j - Q(s_j, a_j; \theta))^2$ w.r.t. $\theta$
      \STATE Reset $\theta^{-} = \theta$ every $C$ steps.
    \ENDFOR
  \ENDFOR
\end{algorithmic}
\end{algorithm}

\begin{table*}
  \centering
  \caption{Parameters used to solve Lunar Lander.}
  \label{tbl:params}
\begin{tabular}{*4l} \toprule
\emph{Name}       & \emph{Symbol} & \emph{Value} & \emph{Description} \\ \midrule
Max episode length                 & $T$ & 400 & \\
Discount factor & $\gamma$ & 0.99   & \\
Replay buffer size & & 100000 & Number of ($s,a,r,s'$) tuples stored \\
Episodes before updates & & 20 & Episodes with uniform policy before training \\
Target network update frequency & $C$ & 1000 & \\
Learning rate & $\alpha$ & 0.0002 & \\
Initial exploration & $\varepsilon_0$ & 1.0 & \\
Final exploration & $\varepsilon_{\text{final}}$ & 0.01 & \\
Final exploration episode & & 1000 & Episodes with linearly decaying exploration \\ \bottomrule
\end{tabular}
\end{table*}

\section{Results}
\label{results}
Figure with total and average reward vs episodes.

\section{Conclusion}
\label{conclusion}
Neural network architecture important. Too simple: Insufficient complexity in representation of all actions, too complex: Training too slow/not stable.


% In the unusual situation where you want a paper to appear in the
% references without citing it in the main text, use \nocite

\bibliography{references}
\bibliographystyle{icml2020}

%%%%%%%%%%%%%%%%%%%%%%%%%%%%%%%%%%%%%%%%%%%%%%%%%%%%%%%%%%%%%%%%%%%%%%%%%%%%%%%
%%%%%%%%%%%%%%%%%%%%%%%%%%%%%%%%%%%%%%%%%%%%%%%%%%%%%%%%%%%%%%%%%%%%%%%%%%%%%%%


\end{document}


% This document was modified from the file originally made available by
% Pat Langley and Andrea Danyluk for ICML-2K. This version was created
% by Iain Murray in 2018, and modified by Alexandre Bouchard in
% 2019 and 2020. Previous contributors include Dan Roy, Lise Getoor and Tobias
% Scheffer, which was slightly modified from the 2010 version by
% Thorsten Joachims & Johannes Fuernkranz, slightly modified from the
% 2009 version by Kiri Wagstaff and Sam Roweis's 2008 version, which is
% slightly modified from Prasad Tadepalli's 2007 version which is a
% lightly changed version of the previous year's version by Andrew
% Moore, which was in turn edited from those of Kristian Kersting and
% Codrina Lauth. Alex Smola contributed to the algorithmic style files.
